\documentclass[14pt]{article}
\usepackage{amsmath, amssymb}
\usepackage[left=1cm, right=1cm, top=1cm, bottom=1cm]{geometry}
\begin{document}
\begin{Large}
I have a doubt about the indeterminations that we find in differential calculus.
$\frac{0}{0}$, $\infty^{\infty}$
When you see these expressions as they tend towards certain values, do they always result in undefined situations? Or is there a way to calculate their limits?
In addition, for those who are not indeterminable (as an example 1 raised to infinity) how can it be interpreted if the limit tends to the base that is one and from below or above?

To address your questions about solving undeterminate forms of limits in differential calculus, let's break down each part step by step.

### Undeterminate Forms

#### \(\frac{0}{0}\)

The expression \(\frac{0}{0}\) is an indeterminate form because it represents a situation where both the numerator and the denominator approach zero. This form does not provide enough information to determine the actual limit without further analysis. Here are some methods to evaluate such limits:

1. **Factoring**: If the expressions in the numerator and the denominator can be factored, you might cancel out common factors that cause the indeterminate form.
2. **Rationalizing**: For limits involving square roots, rationalizing the expression can help eliminate the indeterminate form.
3. **L'Hôpital's Rule**: This rule states that if \(\lim_{x \to c} f(x) = 0\) and \(\lim_{x \to c} g(x) = 0\), then:
   \[
   \lim_{x \to c} \frac{f(x)}{g(x)} = \lim_{x \to c} \frac{f'(x)}{g'(x)}
   \]
   provided the limit on the right-hand side exists.

#### \(\infty^{\infty}\)

The expression \(\infty^{\infty}\) is also an indeterminate form because it represents a situation where both the base and the exponent approach infinity. This form does not provide enough information to determine the actual limit without further analysis. Here are some methods to evaluate such limits:

1. **Rewriting the Expression**: Sometimes, rewriting the expression in a different form can help simplify the problem. For example, if you have \(y = f(x)^{g(x)}\), you can take the natural logarithm of both sides:
   \[
   \ln y = g(x) \ln f(x)
   \]
   Then find the limit of \(\ln y\) and exponentiate to get the original limit.

### Other Indeterminate Forms

There are other indeterminate forms such as \(0 \cdot \infty\), \(\frac{\infty}{\infty}\), \(1^{\infty}\), \(0^0\), and \(\infty - \infty\). Each of these can be handled using similar techniques, including L'Hôpital's Rule, algebraic manipulation, or logarithmic properties.

### Example: Solving \(\frac{0}{0}\)

Consider the limit:
\[
\lim_{x \to 0} \frac{\sin x}{x}
\]
This is an indeterminate form of \(\frac{0}{0}\). We can use L'Hôpital's Rule:
\[
\lim_{x \to 0} \frac{\sin x}{x} = \lim_{x \to 0} \frac{\cos x}{1} = \cos(0) = 1
\]

### Example: Solving \(1^{\infty}\)

Consider the limit:
\[
\lim_{x \to \infty} \left(1 + \frac{1}{x}\right)^x
\]
This is an indeterminate form of \(1^{\infty}\). We can rewrite it using the natural logarithm:
\[
y = \left(1 + \frac{1}{x}\right)^x \implies \ln y = x \ln \left(1 + \frac{1}{x}\right)
\]
Now, find the limit of \(\ln y\):
\[
\lim_{x \to \infty} \ln y = \lim_{x \to \infty} x \ln \left(1 + \frac{1}{x}\right)
\]
Using L'Hôpital's Rule on the indeterminate form \(\frac{\ln \left(1 + \frac{1}{x}\right)}{\frac{1}{x}}\):
\[
\lim_{x \to \infty} x \ln \left(1 + \frac{1}{x}\right) = \lim_{x \to \infty} \frac{\ln \left(1 + \frac{1}{x}\right)}{\frac{1}{x}} = \lim_{x \to \infty} \frac{\frac{-1}{x^2(1 + \frac{1}{x})}}{-\frac{1}{x^2}} = \lim_{x \to \infty} \frac{1}{1 + \frac{1}{x}} = 1
\]
Thus, \(\ln y \to 1\) as \(x \to \infty\), so \(y \to e\). Therefore:
\[
\lim_{x \to \infty} \left(1 + \frac{1}{x}\right)^x = e
\]

### Conclusion

To solve indeterminate forms, it is crucial to use appropriate techniques such as factoring, rationalizing, L'Hôpital's Rule, or rewriting the expression. The specific method depends on the form and the functions involved.

\boxed{\text{Use appropriate methods to evaluate limits of indeterminate forms.}}
\end{Large}
\end{document}